\hypertarget{solidity}{%
\section{Solidity}\label{solidity}}

\hypertarget{spearbot-nodeput_files_to_audit_heresolidityethroutersol}{%
\subsection{spearbot-node/put\_files\_to\_audit\_here/solidity/EthRouter.sol}\label{spearbot-nodeput_files_to_audit_heresolidityethroutersol}}

Summary: The Eth Router contract enables routing of buy, sell, and
change orders for Non-Fungible Tokens (NFTs) to multiple pools in a
single transaction. It supports native ETH as the base token and uses
imported libraries for functionality. The smart contract handles
transactions involving NFTs and private pools, with functions for
selling, depositing, and executing changes. It includes checks for
deadlines, price ranges, and output amounts to ensure valid
transactions. Users can opt to pay royalties for orders going to public
pools.

\hypertarget{openzeppelin-solmate-interface}{%
\subsubsection{Openzeppelin Solmate
Interface}\label{openzeppelin-solmate-interface}}

The content provided is a code snippet written in Solidity, a
programming language used for implementing smart contracts on the
Ethereum blockchain. This code snippet imports two libraries and an
interface, which are used to create and manage Non-Fungible Tokens
(NFTs) and handle royalty payments.

\begin{enumerate}
\def\labelenumi{\arabic{enumi}.}
\item
  IERC2981 Interface: The code imports the IERC2981 interface from the
  OpenZeppelin library. OpenZeppelin is a widely-used library of secure
  and audited smart contracts for the Ethereum blockchain. The IERC2981
  interface is an implementation of the ERC-2981 standard, which is a
  proposal for handling royalty payments for NFTs. This standard allows
  NFT creators to receive a percentage of the sales made on secondary
  markets.
\item
  ERC721 and ERC721TokenReceiver Contracts: The code imports the ERC721
  and ERC721TokenReceiver contracts from the Solmate library. Solmate is
  another library of smart contracts for the Ethereum blockchain. The
  ERC721 contract is an implementation of the ERC-721 standard, which is
  the most widely-used standard for creating and managing NFTs. The
  ERC721TokenReceiver contract is an interface that ensures the
  recipient of an NFT can handle and manage the token correctly.
\item
  SafeTransferLib Library: The code imports the SafeTransferLib library
  from the Solmate library. This library provides utility functions for
  safely transferring tokens between addresses. It ensures that the
  recipient address can handle the token being transferred, preventing
  accidental loss of tokens.
\end{enumerate}

In summary, this code snippet imports the necessary components for
creating and managing NFTs, as well as handling royalty payments for NFT
creators. It uses the IERC2981 interface from the OpenZeppelin library
for royalty payments, the ERC721 and ERC721TokenReceiver contracts from
the Solmate library for NFT creation and management, and the
SafeTransferLib library from Solmate for safely transferring tokens.

\hypertarget{private-pool-registry}{%
\subsubsection{Private Pool Registry}\label{private-pool-registry}}

The given content is a snippet of Solidity code, which is a programming
language used for implementing smart contracts on the Ethereum
blockchain. This code imports various interfaces and contracts from
different libraries and modules to be used in the development of a smart
contract.

\begin{enumerate}
\def\labelenumi{\arabic{enumi}.}
\item
  \texttt{import\ \{IERC2981\}\ from\ "openzeppelin/interfaces/IERC2981.sol";}

  This line imports the IERC2981 interface from the OpenZeppelin
  library. OpenZeppelin is a widely used library for secure smart
  contract development on the Ethereum blockchain. The IERC2981
  interface is an implementation of the ERC-2981 standard, which is a
  royalty standard for Non-Fungible Tokens (NFTs). This standard allows
  NFT creators to receive royalties for secondary sales of their tokens.
\item
  \texttt{import\ \{Pair,\ ReservoirOracle\}\ from\ "caviar/Pair.sol";}

  This line imports two contracts, Pair and ReservoirOracle, from the
  Caviar library's Pair.sol file. Pair is a contract that represents a
  liquidity pool for two tokens, while ReservoirOracle is a contract
  that provides an oracle service for the liquidity pool. An oracle is a
  service that provides external data to smart contracts, and in this
  case, it would provide data related to the liquidity pool.
\item
  \texttt{import\ \{IRoyaltyRegistry\}\ from\ "royalty-registry-solidity/IRoyaltyRegistry.sol";}

  This line imports the IRoyaltyRegistry interface from the Royalty
  Registry Solidity library. The Royalty Registry is a smart contract
  that keeps track of royalty information for NFTs. By implementing this
  interface, the smart contract can interact with the Royalty Registry
  to manage and retrieve royalty information for NFTs.
\item
  \texttt{import\ \{PrivatePool\}\ from\ "./PrivatePool.sol";}

  This line imports the PrivatePool contract from the local file
  PrivatePool.sol. The PrivatePool contract is not described in the
  given content, but based on its name, it could be a contract that
  manages a private liquidity pool or a pool with restricted access.
\end{enumerate}

In summary, this code snippet imports various interfaces and contracts
related to NFT royalties, liquidity pools, and oracles from different
libraries and modules. These imported components can be used to develop
a smart contract that manages NFT royalties and interacts with liquidity
pools and oracles on the Ethereum blockchain.

\hypertarget{stolen-nft-oracle}{%
\subsubsection{Stolen NFT Oracle}\label{stolen-nft-oracle}}

The given content is a single line of code written in Solidity, a
programming language used for implementing smart contracts on the
Ethereum blockchain. This line of code is importing an interface called
\texttt{IStolenNftOracle} from a file named
\texttt{IStolenNftOracle.sol}.

An interface in Solidity is a collection of function signatures that a
contract must implement. It is a way to define the structure and
behavior that a contract should have, without providing the actual
implementation. Other contracts can then inherit and implement this
interface, ensuring that they adhere to the specified structure and
behavior.

In this case, the \texttt{IStolenNftOracle} interface is likely defining
the structure and behavior for an oracle that deals with stolen
non-fungible tokens (NFTs). An oracle is a service that provides
external data to smart contracts on the blockchain. Since blockchains
are deterministic and cannot access external data directly, oracles are
used to bridge the gap between the blockchain and the outside world.

The \texttt{IStolenNftOracle.sol} file is expected to contain the
definition of the \texttt{IStolenNftOracle} interface, which may include
function signatures related to querying, reporting, and validating
stolen NFTs. By importing this interface, the current contract can
implement the required functions and interact with the stolen NFT oracle
in a standardized way.

\hypertarget{eth-router-contract}{%
\subsubsection{Eth Router Contract}\label{eth-router-contract}}

The EthRouter contract, authored by out.eth, is designed to route buy,
sell, and change orders to multiple pools in a single transaction. The
orders can be directed to either a private or a public pool. If an order
is sent to a public pool, users have the option to pay royalties. The
only supported base token is native ETH.

The contract utilizes the ERC721TokenReceiver and SafeTransferLib
libraries. It defines three structs: Buy, Sell, and Change. The Buy
struct contains fields such as pool, NFT, tokenIds, tokenWeights, proof,
baseTokenAmount, and isPublicPool. The Sell struct has similar fields,
with the addition of stolenNftProofs and publicPoolProofs. The Change
struct includes inputTokenIds, inputTokenWeights, inputProof,
stolenNftProofs, outputTokenIds, outputTokenWeights, and outputProof.

The contract also defines several error messages, such as
DeadlinePassed, OutputAmountTooSmall, PriceOutOfRange, and
InvalidRoyaltyFee. The royaltyRegistry address is set as an immutable
public variable.

The contract includes a receive() function to accept external payments.

\hypertarget{royalty-registry-constructor}{%
\subsubsection{Royalty Registry
Constructor}\label{royalty-registry-constructor}}

The given code snippet is a constructor function in Solidity, which is a
programming language used for writing smart contracts on the Ethereum
blockchain. The constructor function is a special function that is
called only once when a smart contract is deployed. It is used to
initialize the state variables of the contract.

In this specific constructor, there is one input parameter:
\texttt{\_royaltyRegistry} of type \texttt{address}. The
\texttt{address} type in Solidity is used to store Ethereum addresses,
which are 20-byte identifiers that represent an account on the Ethereum
network.

The purpose of this constructor is to initialize the state variable
\texttt{royaltyRegistry} with the value of the input parameter
\texttt{\_royaltyRegistry}. The \texttt{royaltyRegistry} variable is
likely a state variable of the contract, which means it is stored on the
blockchain and can be accessed and modified by the contract's functions.

In summary, this constructor function takes an Ethereum address as an
input parameter and initializes the \texttt{royaltyRegistry} state
variable with the provided address. This is likely used to set up a
reference to another smart contract or account that manages royalties
within the context of the contract being deployed.

\hypertarget{buy-operations-execution}{%
\subsubsection{Buy Operations
Execution}\label{buy-operations-execution}}

The given code snippet is a smart contract function called \texttt{buy}
that executes a series of buy operations against public or private pools
in a decentralized marketplace. The function takes three parameters: an
array of \texttt{Buy} objects, a \texttt{deadline} timestamp, and a
boolean flag \texttt{payRoyalties}.

The \texttt{Buy} object contains information about the pool, the NFT
(Non-Fungible Token) being purchased, the base token amount, and other
relevant data. The \texttt{deadline} parameter is used to ensure that
the transaction is mined before the specified timestamp, otherwise, the
transaction will revert. If the deadline is set to 0, there is no
deadline. The \texttt{payRoyalties} flag indicates whether royalties
should be paid or not.

The function first checks if the deadline has passed (if any) and
reverts the transaction if it has. It then iterates through the
\texttt{buys} array and executes each buy operation. If the buy is
against a public pool, it calculates the input amount and pays royalties
if the buyer has opted-in. The royalty fee and recipient are fetched,
and if the royalty fee is greater than 0, it is transferred to the
royalty recipient.

If the buy is against a private pool, the function executes the buy
operation using the provided token weights and proof. After each buy
operation, the NFT is transferred to the caller (buyer).

Finally, any surplus ETH (Ether) remaining in the contract is refunded
to the caller.

\hypertarget{sell-operations-execution}{%
\subsubsection{Sell Operations
Execution}\label{sell-operations-execution}}

The given code defines a \texttt{sell} function that executes a series
of sell operations against public or private pools in a decentralized
exchange. The function takes four parameters: an array of sell
operations (\texttt{sells}), a minimum output amount of tokens
(\texttt{minOutputAmount}), a deadline for the transaction to be mined
(\texttt{deadline}), and a boolean flag indicating whether to pay
royalties or not (\texttt{payRoyalties}).

First, the function checks if the deadline has passed, and if so, it
reverts the transaction with a \texttt{DeadlinePassed} error. Then, it
iterates through the \texttt{sells} array and performs the following
steps for each sell operation:

\begin{enumerate}
\def\labelenumi{\arabic{enumi}.}
\tightlist
\item
  Transfers the Non-Fungible Tokens (NFTs) from the caller to the router
  contract.
\item
  Approves the pool to transfer NFTs from the router.
\item
  If the pool is a public pool, it executes the sell operation against
  the public pool and calculates the output amount. If the seller has
  opted to pay royalties, it calculates the royalty fee and recipient
  for each NFT and transfers the royalty fee to the recipient.
\item
  If the pool is a private pool, it executes the sell operation against
  the private pool.
\end{enumerate}

After all sell operations are executed, the function checks if the
output amount is greater than the minimum specified. If not, it reverts
the transaction with an \texttt{OutputAmountTooSmall} error. Finally, it
transfers the output amount to the caller.

\hypertarget{private-pool-deposit}{%
\subsubsection{Private Pool Deposit}\label{private-pool-deposit}}

The given code snippet is a function called \texttt{deposit} that allows
a user to deposit Non-Fungible Tokens (NFTs) and Ether (ETH) into a
private pool. The function takes the following parameters:

\begin{enumerate}
\def\labelenumi{\arabic{enumi}.}
\tightlist
\item
  \texttt{privatePool}: The address of the private pool to deposit to.
\item
  \texttt{nft}: The contract address of the NFT.
\item
  \texttt{tokenIds}: An array of token IDs of the NFTs to be deposited.
\item
  \texttt{minPrice}: The minimum price of the pool. The function will
  revert if the pool's price is smaller than this value.
\item
  \texttt{maxPrice}: The maximum price of the pool. The function will
  revert if the pool's price is greater than this value.
\item
  \texttt{deadline}: The deadline for the transaction to be mined. The
  function will revert if the current timestamp is greater than the
  deadline. If set to 0, the deadline will be ignored.
\end{enumerate}

The function first checks if the deadline has passed (if any) and
reverts with a \texttt{DeadlinePassed} error if the condition is met.
Next, it checks if the pool's price is within the specified range
(between \texttt{minPrice} and \texttt{maxPrice}) and reverts with a
\texttt{PriceOutOfRange} error if the condition is not met.

The function then transfers the specified NFTs from the caller to the
contract address using the \texttt{safeTransferFrom} function of the
ERC721 standard. After that, it approves the private pool to transfer
NFTs from the router by calling the \texttt{setApprovalForAll} function
of the ERC721 standard.

Finally, the function executes the deposit by calling the
\texttt{deposit} function of the \texttt{PrivatePool} contract, passing
the \texttt{tokenIds} array and the Ether value sent with the
transaction.

\hypertarget{pool-change-execution}{%
\subsubsection{Pool Change Execution}\label{pool-change-execution}}

The given code snippet is a Solidity function called \texttt{change}
that executes a series of change operations against a private pool in a
decentralized application. The function takes two input parameters: an
array of \texttt{Change} objects called \texttt{changes} and a
\texttt{deadline} represented as a uint256.

The function first checks if the deadline has passed by comparing the
current block timestamp with the given deadline. If the deadline has
passed and is not set to 0 (which means it should be ignored), the
function reverts with a \texttt{DeadlinePassed} error.

Next, the function iterates through the \texttt{changes} array and
processes each change operation. For each change, it transfers the input
NFTs (Non-Fungible Tokens) from the caller to the contract's address
using the \texttt{safeTransferFrom} function of the ERC721 standard. It
then approves the private pool to transfer NFTs from the router by
calling the \texttt{setApprovalForAll} function of the ERC721 standard.

The function then executes the change operation by calling the
\texttt{change} function of the \texttt{PrivatePool} contract, passing
in the necessary parameters such as input and output token IDs, weights,
and proofs.

After the change operation is executed, the function transfers the
output NFTs back to the caller using the \texttt{safeTransferFrom}
function of the ERC721 standard.

Finally, if there is any remaining ETH balance in the contract, it is
refunded to the caller using the \texttt{safeTransferETH} function.

\hypertarget{royalty-fee-recipient}{%
\subsubsection{Royalty Fee Recipient}\label{royalty-fee-recipient}}

The given content is a function definition in a smart contract, written
in Solidity programming language, for a blockchain-based application.
The function is named \texttt{getRoyalty} and its purpose is to
calculate the royalty fee and identify the recipient for a given
Non-Fungible Token (NFT) and its sale price. The function retrieves the
royalty information from the manifold registry.

The function accepts three input parameters:

\begin{enumerate}
\def\labelenumi{\arabic{enumi}.}
\tightlist
\item
  \texttt{address\ nft}: The address of the NFT contract.
\item
  \texttt{uint256\ tokenId}: The unique identifier of the NFT.
\item
  \texttt{uint256\ salePrice}: The sale price of the NFT.
\end{enumerate}

The function returns two output values:

\begin{enumerate}
\def\labelenumi{\arabic{enumi}.}
\tightlist
\item
  \texttt{uint256\ royaltyFee}: The calculated royalty fee to be paid.
\item
  \texttt{address\ recipient}: The address of the recipient who will
  receive the royalty fee.
\end{enumerate}

The function is marked as \texttt{public} and \texttt{view}, which means
it can be called by any external entity and does not modify the state of
the contract.

Inside the function, the first step is to get the royalty lookup address
by calling the \texttt{getRoyaltyLookupAddress} function of the
\texttt{IRoyaltyRegistry} interface, passing the NFT contract address as
an argument. The \texttt{royaltyRegistry} variable is used to store the
address of the royalty registry contract.

The function then returns the calculated royalty fee and the recipient
address based on the information retrieved from the manifold registry.

\hypertarget{royalty-fee-validation}{%
\subsubsection{Royalty Fee Validation}\label{royalty-fee-validation}}

The given code snippet is written in Solidity, a programming language
used for implementing smart contracts on the Ethereum blockchain. It
checks if a specific contract, identified by its address
(lookupAddress), supports the ERC-2981 interface, which is a standard
for handling royalties in Non-Fungible Tokens (NFTs). If the contract
supports this interface, the code retrieves the royalty information for
a specific token (tokenId) at a given sale price (salePrice) and checks
if the royalty fee is valid.

\begin{enumerate}
\def\labelenumi{\arabic{enumi}.}
\item
  The first line checks if the contract at the lookupAddress supports
  the ERC-2981 interface by calling the supportsInterface function with
  the interfaceId of the IERC2981 type. If the contract supports the
  interface, the code proceeds to the next step.
\item
  The second line retrieves the royalty information for the specified
  tokenId and salePrice by calling the royaltyInfo function of the
  IERC2981 interface. The function returns two values: the recipient of
  the royalty fee (recipient) and the amount of the royalty fee
  (royaltyFee).
\item
  The third line checks if the royaltyFee is greater than the salePrice.
  If it is, the code reverts the transaction and throws an
  InvalidRoyaltyFee error. This ensures that the royalty fee cannot be
  higher than the sale price of the token.
\item
  If the royalty fee is valid, the code proceeds with the rest of the
  smart contract logic (not shown in the snippet).
\end{enumerate}

In summary, this code snippet is used to verify if a contract supports
the ERC-2981 royalty standard, retrieve the royalty information for a
specific NFT, and ensure that the royalty fee is valid before proceeding
with the rest of the smart contract logic.

\hypertarget{spearbot-nodeput_files_to_audit_heresolidityfactorysol}{%
\subsection{spearbot-node/put\_files\_to\_audit\_here/solidity/Factory.sol}\label{spearbot-nodeput_files_to_audit_heresolidityfactorysol}}

Summary: The Caviar Private Pool Factory is a smart contract used to
create and initialize new private pools. Each time a private pool is
created, an NFT representing that pool is minted to the creator. The
contract also handles protocol fees, which accrue to the contract and
can be withdrawn by the admin. The Factory contract uses the minimal
proxy pattern to deploy private pool clones, and allows for setting and
updating private pool metadata, implementation contracts, and protocol
fee rates. Additionally, it provides functionality for withdrawing
earned protocol fees and predicting deployment addresses of new private
pools.

\hypertarget{caviar-pool-factory}{%
\subsubsection{Caviar Pool Factory}\label{caviar-pool-factory}}

The Caviar Private Pool Factory is a smart contract that facilitates the
creation and initialization of new private pools. It is an ERC721
contract, which means that each time a private pool is created, a unique
non-fungible token (NFT) is minted and assigned to the creator of the
pool. The contract also handles the accumulation of protocol fees, which
can be withdrawn by the admin.

The Factory contract utilizes the LibClone library for cloning the
private pool implementation, and the SafeTransferLib library for safely
transferring tokens. It has two main events: Create and Withdraw. The
Create event is emitted when a new private pool is created, and it
includes the private pool's address, token IDs, and the base token
amount. The Withdraw event is emitted when the admin withdraws tokens
from the contract, and it includes the token's address and the withdrawn
amount.

The contract has a public variable, privatePoolImplementation, which
stores the address of the private pool implementation that proxies point
to. This allows for easy upgrades and modifications to the private pool
implementation without affecting the existing private pools.

\hypertarget{private-pool-metadata}{%
\subsubsection{Private Pool Metadata}\label{private-pool-metadata}}

The given code snippet is a Solidity function named
\texttt{setPrivatePoolMetadata} that sets the address of a private pool
metadata contract. This function is part of a smart contract and can
only be executed by the contract owner, as indicated by the
\texttt{onlyOwner} modifier.

The function takes one input parameter, \texttt{\_privatePoolMetadata},
which is an address representing the private pool metadata contract.
Inside the function, the value of the \texttt{\_privatePoolMetadata}
parameter is assigned to the \texttt{privatePoolMetadata} state
variable, effectively updating the address of the private pool metadata
contract.

The \texttt{@notice} and \texttt{@param} comments above the function
provide a brief description of the function's purpose and its input
parameter, respectively. These comments are part of the NatSpec
documentation format, which is used to generate human-readable
descriptions of Solidity code.

\hypertarget{caviar-private-pools}{%
\subsubsection{Caviar Private Pools}\label{caviar-private-pools}}

The given content is a Solidity code snippet representing a constructor
function and a receive function for a smart contract. This smart
contract is based on the Ethereum blockchain and utilizes the ERC721
standard for non-fungible tokens (NFTs). The constructor function
initializes the contract with a name, a symbol, and an owner, while the
receive function allows the contract to accept Ether payments.

\begin{enumerate}
\def\labelenumi{\arabic{enumi}.}
\item
  Constructor function: The constructor function is defined with the
  keyword 'constructor()'. It is a special function that is executed
  only once when the smart contract is deployed on the Ethereum
  blockchain. In this case, the constructor function takes no arguments.

  Inside the constructor, there are three main components:

  a. ERC721: This is a standard interface for non-fungible tokens (NFTs)
  on the Ethereum blockchain. The constructor initializes the contract
  with the given name "Caviar Private Pools" and symbol "POOL". This
  means that the NFTs created by this contract will have this name and
  symbol associated with them.

  b. Owned: This is a contract modifier that ensures that only the owner
  of the contract can execute certain functions. In this case, the
  constructor sets the owner of the contract to be the address that
  deploys the contract (msg.sender). This means that only the deployer
  of the contract will have control over certain functions, as defined
  by the Owned modifier.
\item
  Receive function: The receive function is defined with the keyword
  'receive()' and the 'external payable' modifier. This function allows
  the smart contract to accept Ether payments directly to its address.
  The 'external' keyword means that this function can only be called
  from outside the contract, while the 'payable' keyword allows the
  function to receive Ether.

  In this case, the receive function is empty, meaning that it does not
  perform any specific actions when Ether is sent to the contract's
  address. However, the contract can still store the received Ether and
  use it for various purposes, as defined by other functions in the
  contract.
\end{enumerate}

\hypertarget{private-pool-creation}{%
\subsubsection{Private Pool Creation}\label{private-pool-creation}}

The \texttt{create} function is used to create a new private pool using
the minimal proxy pattern that points to the private pool
implementation. The caller must approve the factory to transfer the NFTs
that will be deposited to the pool. The function takes several
parameters, including the base token address, NFT address, virtual base
token reserves, virtual NFT reserves, change fee, fee rate, Merkle root,
whether to use the stolen NFT oracle, whether to pay royalties, salt,
token IDs, and the base token amount.

The function first checks if the \texttt{msg.value} is equal to the base
token amount if the base token is ETH or if the \texttt{msg.value} is
equal to zero if the base token is not ETH. If the condition is not met,
it reverts with an \texttt{InvalidEthAmount} error.

Next, the function deploys a minimal proxy clone of the private pool
implementation and mints the NFT to the caller. It then initializes the
pool with the provided parameters.

If the base token is ETH, the function transfers ETH into the pool.
Otherwise, it deposits the base tokens from the caller into the pool. It
then deposits the NFTs from the caller into the pool.

Finally, the function emits a \texttt{Create} event with the private
pool address, token IDs, and base token amount. The function returns the
address of the created private pool.

\hypertarget{private-pool-implementation}{%
\subsubsection{Private Pool
Implementation}\label{private-pool-implementation}}

The given code snippet is a Solidity function named
\texttt{setPrivatePoolImplementation} that is used to set the private
pool implementation contract address for newly deployed proxies. This
function takes an input parameter \texttt{\_privatePoolImplementation},
which is the address of the private pool implementation contract. The
function is marked as \texttt{public}, meaning it can be called from any
external contract or account, and it has a modifier \texttt{onlyOwner},
which restricts the execution of the function to the contract owner
only.

Inside the function, the private state variable
\texttt{privatePoolImplementation} is assigned the value of the input
parameter \texttt{\_privatePoolImplementation}. This effectively updates
the private pool implementation contract address that will be used by
newly deployed proxies.

\hypertarget{set-protocol-fee}{%
\subsubsection{Set Protocol Fee}\label{set-protocol-fee}}

The given code snippet is a Solidity function named
\texttt{setProtocolFeeRate} that sets the protocol fee rate for a
specific smart contract. This function is designed to be called only by
the contract owner, as indicated by the \texttt{onlyOwner} modifier.

The protocol fee rate is expressed in basis points, where 1 basis point
is equal to 0.01\%. For example, a value of 350 basis points represents
a 3.5\% fee rate. The function takes a single input parameter,
\texttt{\_protocolFeeRate}, which is a 16-bit unsigned integer
representing the desired protocol fee rate in basis points.

The function sets the value of the contract's \texttt{protocolFeeRate}
state variable to the input parameter \texttt{\_protocolFeeRate}. This
state variable is then used by other functions within the contract to
calculate the protocol fee for various operations such as buy, sell, or
change.

The \texttt{@notice} and \texttt{@param} comments provide additional
information about the function's purpose and input parameters, which can
be used by developers and tools to generate documentation or provide
contextual information while working with the code.

\hypertarget{withdraw-token-amount}{%
\subsubsection{Withdraw Token Amount}\label{withdraw-token-amount}}

The given code snippet is a Solidity function named \texttt{withdraw}
that allows the owner of a smart contract to withdraw earned protocol
fees in the form of a specified token and amount. The function takes two
input parameters: \texttt{token} (an address representing the token to
be withdrawn) and \texttt{amount} (a uint256 value representing the
amount of tokens to be withdrawn).

The function has a \texttt{public} visibility, meaning it can be called
from any external account or contract, and it has a \texttt{onlyOwner}
modifier, which restricts its execution to the contract owner only.

Inside the function, there is a conditional statement that checks if the
provided \texttt{token} address is equal to the zero address
(address(0)). If it is, the function assumes that the withdrawal is for
Ether (ETH) and calls the \texttt{safeTransferETH} function from the
\texttt{msg.sender} (the contract owner) with the specified
\texttt{amount}. If the \texttt{token} address is not the zero address,
the function assumes that the withdrawal is for an ERC20 token and calls
the \texttt{transfer} function from the ERC20 token contract with the
\texttt{msg.sender} and the specified \texttt{amount}.

After the withdrawal is executed, the function emits an event named
\texttt{Withdraw} with the \texttt{token} and \texttt{amount} as its
arguments. This event can be used by external services or applications
to track and monitor withdrawals from the smart contract.

\hypertarget{token-uri-function}{%
\subsubsection{Token URI Function}\label{token-uri-function}}

The given code snippet is a function named \texttt{tokenURI} written in
Solidity, which is a programming language used for implementing smart
contracts on the Ethereum blockchain. This function is marked as
\texttt{public}, \texttt{view}, and \texttt{override}, meaning it can be
called from outside the contract, does not modify the contract's state,
and overrides a function with the same name in a parent contract.

The \texttt{tokenURI} function takes a single input parameter,
\texttt{id}, which is a 256-bit unsigned integer representing the token
ID. The purpose of this function is to return the token URI associated
with the given token ID. A token URI is typically a URL that points to a
JSON file containing metadata about the token, such as its name,
description, and image.

Inside the function, it calls another function named \texttt{tokenURI}
from a contract named \texttt{PrivatePoolMetadata}. The contract address
for \texttt{PrivatePoolMetadata} is passed as an argument to the
function. This suggests that the \texttt{PrivatePoolMetadata} contract
is responsible for managing the token URIs for the tokens in the private
pool.

The \texttt{tokenURI} function in the \texttt{PrivatePoolMetadata}
contract is expected to return a string in memory, which is then
returned by the \texttt{tokenURI} function in the current contract. This
allows users or other contracts to query the token URI for a specific
token ID in the private pool.

\hypertarget{predict-pool-address}{%
\subsubsection{Predict Pool Address}\label{predict-pool-address}}

The given code snippet is a function named
\texttt{predictPoolDeploymentAddress} that predicts the deployment
address of a new private pool in a blockchain-based system. This
function takes a single input parameter, \texttt{salt}, which is a
32-byte value used during the deployment process. The function returns a
single output, \texttt{predictedAddress}, which is the predicted
deployment address of the private pool.

The function is marked as \texttt{public}, meaning it can be called by
any external entity, and \texttt{view}, which indicates that it does not
modify the state of the blockchain. This function is part of a smart
contract, which is a self-executing contract with the terms of the
agreement directly written into code.

Inside the function, the \texttt{predictedAddress} is calculated by
calling the \texttt{predictDeterministicAddress} function on the
\texttt{privatePoolImplementation} object, passing the \texttt{salt} and
the address of the current smart contract instance (represented by
\texttt{address(this)}) as arguments. The
\texttt{predictDeterministicAddress} function is responsible for
generating the deterministic address based on the provided salt and the
current contract address.

In summary, the \texttt{predictPoolDeploymentAddress} function is a
public view function that predicts the deployment address of a new
private pool using a given salt value. It does so by calling the
\texttt{predictDeterministicAddress} function on the
\texttt{privatePoolImplementation} object with the salt and the current
contract address as arguments.

\hypertarget{spearbot-nodeput_files_to_audit_heresolidityistolennftoraclesol}{%
\subsection{spearbot-node/put\_files\_to\_audit\_here/solidity/IStolenNftOracle.sol}\label{spearbot-nodeput_files_to_audit_heresolidityistolennftoraclesol}}

Summary: The IStolenNftOracle interface in Solidity defines a structure
for messages and a function to validate that a set of token IDs have not
been marked as stolen by the oracle. The function takes the token
contract address, token IDs, and proofs as input parameters.

\hypertarget{stolen-nft-oracle-1}{%
\subsubsection{Stolen NFT Oracle}\label{stolen-nft-oracle-1}}

The given content represents an interface called
\texttt{IStolenNftOracle} in a programming context, most likely within a
blockchain or smart contract environment. An interface is a collection
of abstract methods (functions) that can be implemented by any class or
contract that chooses to use it. In this case, the interface is
specifically designed for handling stolen Non-Fungible Tokens (NFTs)
within an oracle system.

An oracle, in the context of blockchain and smart contracts, is a system
that provides external data to smart contracts. This data can be used to
trigger specific actions or decisions within the contract. In the case
of \texttt{IStolenNftOracle}, the oracle would provide information about
stolen NFTs, which can be used by other smart contracts or applications
to make decisions based on the status of the NFTs.

The \texttt{IStolenNftOracle} interface does not provide any
implementation details or specific methods. However, it serves as a
blueprint for developers to create their own implementations of the
interface, which would include methods for querying and updating the
oracle with information about stolen NFTs.

Some possible methods that could be included in an implementation of the
\texttt{IStolenNftOracle} interface are:

\begin{enumerate}
\def\labelenumi{\arabic{enumi}.}
\item
  \texttt{isStolen}: A method that takes an NFT identifier as input and
  returns a boolean value indicating whether the NFT is marked as stolen
  in the oracle.
\item
  \texttt{reportStolen}: A method that allows users to report an NFT as
  stolen by providing its identifier and additional information, such as
  the theft date and the original owner's address.
\item
  \texttt{removeStolen}: A method that allows authorized users (e.g.,
  the original owner or an administrator) to remove an NFT from the list
  of stolen NFTs in the oracle.
\item
  \texttt{getStolenNftData}: A method that retrieves detailed
  information about a stolen NFT, such as its identifier, theft date,
  and original owner's address.
\end{enumerate}

By implementing the \texttt{IStolenNftOracle} interface, developers can
create a standardized way for smart contracts and applications to
interact with an oracle that provides information about stolen NFTs.
This can help improve the security and transparency of NFT marketplaces
and other platforms that deal with NFTs.

\hypertarget{message-signature-timestamp}{%
\subsubsection{Message Signature
Timestamp}\label{message-signature-timestamp}}

The given content is a Solidity code snippet that defines a struct
called "Message" within a smart contract. Solidity is a programming
language used for writing smart contracts on the Ethereum blockchain.

The "Message" struct consists of four fields:

\begin{enumerate}
\def\labelenumi{\arabic{enumi}.}
\item
  \texttt{id}: A bytes32 variable that represents the unique identifier
  of the message. The \texttt{bytes32} type is an array of 32 bytes,
  which is commonly used for storing fixed-size hashes or other data
  that needs to be compact and efficient.
\item
  \texttt{payload}: A bytes variable that holds the actual content of
  the message. The \texttt{bytes} type is a dynamically-sized byte
  array, which means it can store an arbitrary amount of data.
\item
  \texttt{timestamp}: A uint256 variable that stores the UNIX timestamp
  when the message was signed by the oracle. The \texttt{uint256} type
  is an unsigned 256-bit integer, which can store large numbers and is
  commonly used for timestamps and other numerical data in smart
  contracts.
\item
  \texttt{signature}: A bytes variable that contains the ECDSA signature
  or EIP-2098 compact signature of the message. This signature is used
  to verify the authenticity of the message, ensuring that it was indeed
  signed by the oracle and has not been tampered with.
\end{enumerate}

In summary, the "Message" struct is a data structure used to store
information about a message signed by an oracle in a smart contract. It
includes a unique identifier, the message payload, a timestamp
indicating when the message was signed, and a cryptographic signature to
ensure the message's authenticity.

\hypertarget{validate-not-stolen}{%
\subsubsection{Validate Not Stolen}\label{validate-not-stolen}}

The given content describes a function called
\texttt{validateTokensAreNotStolen} in a smart contract. This function
is responsible for validating that a set of token ids have not been
marked as stolen by an oracle. An oracle is an external data source that
provides information to smart contracts.

The function takes three input parameters:

\begin{enumerate}
\def\labelenumi{\arabic{enumi}.}
\tightlist
\item
  \texttt{tokenAddress}: This is the address of the token contract,
  which is a unique identifier for the token on the blockchain.
\item
  \texttt{tokenIds}: This is an array of token ids that need to be
  validated. Each token id is a unique identifier for a specific token.
\item
  \texttt{proofs}: This is an array of \texttt{Message} objects, which
  are the proofs that the token ids have not been marked as stolen.
  These proofs are signed messages from the oracle.
\end{enumerate}

The function is marked as \texttt{external}, which means it can only be
called from outside the smart contract. The purpose of this function is
to check the signed messages from the oracle to ensure that the given
token ids have not been marked as stolen. This validation process is
crucial for maintaining the integrity and security of the token
ecosystem.

\hypertarget{spearbot-nodeput_files_to_audit_heresolidityprivatepoolsol}{%
\subsection{spearbot-node/put\_files\_to\_audit\_here/solidity/PrivatePool.sol}\label{spearbot-nodeput_files_to_audit_heresolidityprivatepoolsol}}

Summary: The Private Pool is a single-owner, customizable NFT Automated
Market Maker (AMM) smart contract that offers concentrated liquidity,
custom fee rates, stolen NFT filtering, custom NFT weightings, royalty
support, and flash loans. Users can create a pool, deposit NFTs and base
tokens, enable trading, and earn fees on each trade. The contract
includes functions for buying, selling, depositing, and withdrawing
NFTs, as well as setting virtual reserves, fee rates, and other
configurations. It also supports royalty payments to NFT creators and
checks for stolen NFTs using an oracle. The code defines functions for
calculating fees, input and output amounts, and prices for buying,
selling, and changing NFTs in a pool, as well as handling flash loans
for NFTs.

\hypertarget{openzeppelin-solmate-interface-1}{%
\subsubsection{OpenZeppelin Solmate
Interface}\label{openzeppelin-solmate-interface-1}}

The content provided is a code snippet written in Solidity, a
programming language used for implementing smart contracts on the
Ethereum blockchain. The code imports various libraries and interfaces
that are commonly used in the development of smart contracts.

\begin{enumerate}
\def\labelenumi{\arabic{enumi}.}
\item
  \texttt{import\ \{IERC2981\}\ from\ "openzeppelin/interfaces/IERC2981.sol";}:
  This line imports the IERC2981 interface from the OpenZeppelin
  library. IERC2981 is an interface for the ERC-2981 standard, which is
  a royalty standard for Non-Fungible Tokens (NFTs). It allows NFT
  creators to receive royalties for secondary sales of their tokens.
\item
  The ASCII art included in the code snippet is a decorative element and
  does not have any functional impact on the code.
\item
  \texttt{import\ \{ERC20\}\ from\ "solmate/tokens/ERC20.sol";}: This
  line imports the ERC20 token implementation from the Solmate library.
  ERC20 is a widely-used standard for creating and managing fungible
  tokens on the Ethereum blockchain.
\item
  \texttt{import\ \{ERC721,\ ERC721TokenReceiver\}\ from\ "solmate/tokens/ERC721.sol";}:
  This line imports the ERC721 token implementation and the
  ERC721TokenReceiver interface from the Solmate library. ERC721 is a
  standard for creating and managing non-fungible tokens (NFTs) on the
  Ethereum blockchain, while the ERC721TokenReceiver interface is used
  to ensure that a contract can correctly handle ERC721 tokens.
\item
  \texttt{import\ \{FixedPointMathLib\}\ from\ "solmate/utils/FixedPointMathLib.sol";}:
  This line imports the FixedPointMathLib library from the Solmate
  library. FixedPointMathLib is a utility library that provides
  functions for performing fixed-point arithmetic operations, which are
  useful for handling decimal numbers in smart contracts.
\item
  \texttt{import\ \{SafeTransferLib\}\ from\ "solmate/utils/SafeTransferLib.sol";}:
  This line imports the SafeTransferLib library from the Solmate
  library. SafeTransferLib is a utility library that provides functions
  for safely transferring ERC20 and ERC721 tokens, preventing common
  issues such as reentrancy attacks and improperly handling token
  transfers.
\item
  \texttt{import\ \{MerkleProofLib\}\ from\ "solady/utils/MerkleProofLib.sol";}:
  This line imports the MerkleProofLib library from the Solady library.
  MerkleProofLib is a utility library that provides functions for
  working with Merkle proofs, which are used to verify the membership of
  an element in a Merkle tree. This is useful for various applications,
  such as verifying the authenticity of data or proving the ownership of
  a token.
\end{enumerate}

\hypertarget{royalty-registry-interface}{%
\subsubsection{Royalty Registry
Interface}\label{royalty-registry-interface}}

The given content is a code snippet written in Solidity, a programming
language used for implementing smart contracts on the Ethereum
blockchain. This code snippet imports two interfaces, IERC2981 and
IRoyaltyRegistry, from their respective source files.

\begin{enumerate}
\def\labelenumi{\arabic{enumi}.}
\item
  IERC2981: This interface is imported from the OpenZeppelin library, a
  widely-used framework for secure smart contract development. IERC2981
  is an interface for the ERC-2981 standard, which is a proposed
  standard for handling royalty payments for Non-Fungible Tokens (NFTs)
  on the Ethereum blockchain. The ERC-2981 standard aims to provide a
  consistent way for NFT creators to receive royalties whenever their
  NFTs are sold or transferred. By implementing this interface, a smart
  contract can support royalty payments according to the ERC-2981
  standard.
\item
  IRoyaltyRegistry: This interface is imported from the Royalty Registry
  Solidity library, which is a collection of smart contracts and
  interfaces designed to manage royalty payments for NFTs. The
  IRoyaltyRegistry interface provides a set of functions that allow a
  smart contract to interact with a royalty registry, which is a central
  repository for storing and managing royalty information for NFTs. By
  implementing this interface, a smart contract can query and update
  royalty information for NFTs in a standardized and efficient manner.
\end{enumerate}

In summary, this code snippet imports two interfaces related to royalty
management for NFTs on the Ethereum blockchain. The IERC2981 interface
provides a standard for handling royalty payments, while the
IRoyaltyRegistry interface allows for interaction with a central
registry for managing royalty information. Implementing these interfaces
in a smart contract enables support for consistent and efficient royalty
management for NFT creators and owners.

\hypertarget{erc3156-flash-interface}{%
\subsubsection{ERC3156 Flash Interface}\label{erc3156-flash-interface}}

The given content is a single line of code that imports the
\texttt{IERC3156FlashBorrower} interface from the OpenZeppelin library's
\texttt{IERC3156FlashLender.sol} file. This line of code is written in
Solidity, a programming language used for implementing smart contracts
on the Ethereum blockchain.

OpenZeppelin is a widely-used library of secure and audited smart
contract components for the Ethereum platform. It provides developers
with reusable components to build decentralized applications, ensuring
that the code is secure and follows best practices.

The \texttt{IERC3156FlashBorrower} interface is part of the ERC-3156
standard, which defines a common interface for flash loans. Flash loans
are a feature in decentralized finance (DeFi) that allows users to
borrow assets without collateral for a very short period, typically
within a single transaction. The borrowed assets must be returned within
the same transaction, or the transaction will be reverted. This feature
is useful for various use cases, such as arbitrage, collateral swapping,
and self-liquidation.

By importing the \texttt{IERC3156FlashBorrower} interface, the developer
can implement the required functions in their smart contract to interact
with flash loan providers that follow the ERC-3156 standard. This
ensures compatibility and interoperability between different flash loan
providers and borrowers in the DeFi ecosystem.

\hypertarget{stolen-nft-oracle-2}{%
\subsubsection{Stolen NFT Oracle}\label{stolen-nft-oracle-2}}

The given code snippet is a Solidity function named
\texttt{setUseStolenNftOracle} that is used to set the flag for using a
stolen NFT oracle in a smart contract. This function can only be called
by the owner of the pool, as indicated by the \texttt{onlyOwner}
modifier.

The stolen NFT oracle is a mechanism that checks if a given NFT
(Non-Fungible Token) is stolen or not. The function takes a boolean
parameter \texttt{newUseStolenNftOracle}, which represents the new flag
value for using the stolen NFT oracle.

Inside the function, the \texttt{useStolenNftOracle} state variable is
assigned the value of the \texttt{newUseStolenNftOracle} parameter. This
updates the flag for using the stolen NFT oracle in the smart contract.

After updating the flag, the function emits an event called
\texttt{SetUseStolenNftOracle} with the new flag value as its argument.
This event can be used by external systems or applications to track
changes in the use of the stolen NFT oracle.

\hypertarget{private-pool-nft}{%
\subsubsection{Private Pool NFT}\label{private-pool-nft}}

The PrivatePool contract is an NFT Automated Market Maker (AMM)
controlled by a single owner with features such as concentrated
liquidity, custom fee rates, stolen NFT filtering, custom NFT
weightings, royalty support, and flash loans. Users can create a pool
and modify its parameters according to their preferences. Depositing
NFTs and base tokens (or ETH) into the pool enables trading, and users
can earn fees on each trade.

The contract includes events for initializing the pool, buying and
selling NFTs, depositing and withdrawing tokens, changing NFTs, and
setting various parameters such as virtual reserves, Merkle root, fee
rate, stolen NFT oracle usage, and royalty payments.

The PrivatePool contract stores information about the base ERC20 token,
the NFT address, change/flash fee, buy/sell fee rate, initialization
status, royalty payment status, stolen NFT oracle usage, virtual base
token reserves, virtual NFT reserves, Merkle root, and the stolen NFT
oracle address.

The contract checks for errors such as unauthorized access, invalid ETH
amounts, invalid Merkle proofs, insufficient input weight, high fee
rates, unavailability for flash loans, failed flash loans, and invalid
royalty fees.

The contract allows users to set virtual reserves for base tokens and
NFTs, which affect the liquidity depth and price of the pool. Users can
also set the Merkle root for custom NFT weightings, and enable or
disable the use of a stolen NFT oracle and royalty payments.

\hypertarget{factory-contract-creator}{%
\subsubsection{Factory Contract
Creator}\label{factory-contract-creator}}

The given content is a snippet of a Solidity smart contract code that
defines a factory contract, a royalty registry, a modifier, and a
receive function.

\begin{enumerate}
\def\labelenumi{\arabic{enumi}.}
\item
  Factory Contract: The code defines an immutable public payable address
  named 'factory'. This address represents the factory contract that
  created the current pool. The 'immutable' keyword indicates that the
  value of this variable cannot be changed after the contract is
  deployed.
\item
  Royalty Registry: The code also defines an immutable public address
  named 'royaltyRegistry'. This address represents the royalty registry
  from manifold.xyz, a platform that helps creators manage royalties for
  their digital assets. Similar to the factory contract, the 'immutable'
  keyword ensures that the value of this variable cannot be changed
  after the contract is deployed.
\item
  Modifier 'onlyOwner': The code defines a modifier named 'onlyOwner'
  that checks if the sender of the transaction (msg.sender) is the owner
  of the contract. The owner is determined by calling the 'ownerOf'
  function from the Factory contract, passing the uint160 representation
  of the current contract's address as an argument. If the sender is not
  the owner, the modifier will revert the transaction with an
  'Unauthorized' error message. The 'virtual' keyword indicates that
  this modifier can be overridden in derived contracts.
\item
  Receive Function: The code includes a receive function, which is an
  unnamed external payable function that allows the contract to accept
  Ether payments. This function is automatically called when the
  contract receives Ether without any data provided.
\end{enumerate}

\hypertarget{immutable-parameters-deployment}{%
\subsubsection{Immutable Parameters
Deployment}\label{immutable-parameters-deployment}}

The given code snippet is a constructor function for a smart contract in
the Solidity programming language. This constructor is called only when
the base implementation contract is deployed. It sets three immutable
parameters: factory, royaltyRegistry, and stolenNftOracle. These
parameters represent the addresses of the factory contract, the royalty
registry from manifold.xyz, and the stolen NFT oracle, respectively.

The constructor takes three input arguments: \_factory,
\_royaltyRegistry, and \_stolenNftOracle, which are the addresses for
the respective contracts. Inside the constructor, the factory address is
converted to a payable address and assigned to the factory variable. The
royaltyRegistry and stolenNftOracle variables are assigned the values of
\_royaltyRegistry and \_stolenNftOracle, respectively.

These parameters are stored in immutable storage, which allows all
minimal proxy contracts to read them without incurring additional
deployment costs and re-initializing them at the point of creation in
the factory contract. Immutable storage is a feature in Solidity that
ensures the values of these variables cannot be changed after the
contract is deployed, providing security and efficiency benefits.

\hypertarget{private-pool-initialization}{%
\subsubsection{Private Pool
Initialization}\label{private-pool-initialization}}

The given code snippet is a function named \texttt{initialize} that sets
up a private pool with its initial parameters. This function should only
be called once by the factory. The function takes the following input
parameters:

\begin{enumerate}
\def\labelenumi{\arabic{enumi}.}
\tightlist
\item
  \texttt{\_baseToken}: The address of the base token.
\item
  \texttt{\_nft}: The address of the NFT (Non-Fungible Token).
\item
  \texttt{\_virtualBaseTokenReserves}: The virtual base token reserves.
\item
  \texttt{\_virtualNftReserves}: The virtual NFT reserves.
\item
  \texttt{\_changeFee}: The change fee.
\item
  \texttt{\_feeRate}: The fee rate (in basis points), where 200 equals
  2\%.
\item
  \texttt{\_merkleRoot}: The Merkle root.
\item
  \texttt{\_useStolenNftOracle}: A boolean value indicating whether the
  pool uses the stolen NFT oracle to check if an NFT is stolen.
\item
  \texttt{\_payRoyalties}: A boolean value indicating whether to pay
  royalties.
\end{enumerate}

The function first checks if the pool has already been initialized, and
if so, it reverts with an \texttt{AlreadyInitialized} error. It then
checks if the fee rate is less than 50\% (5,000 basis points), and if
not, it reverts with a \texttt{FeeRateTooHigh} error.

Next, the function sets the state variables with the input parameters.
It marks the pool as initialized and emits an \texttt{Initialize} event
with the input parameters.

In summary, the \texttt{initialize} function is responsible for setting
up a private pool with its initial parameters, ensuring that it is only
called once and that the fee rate is within acceptable limits. It also
sets the state variables and emits an event to notify listeners of the
pool's initialization.

\hypertarget{nft-pool-purchase}{%
\subsubsection{NFT Pool Purchase}\label{nft-pool-purchase}}

The given content describes a function called "buy" that allows users to
purchase NFTs (Non-Fungible Tokens) from a pool using base tokens. The
function takes three parameters: tokenIds, tokenWeights, and proof. The
tokenIds are the unique identifiers of the NFTs to be purchased,
tokenWeights represent the assigned weights of the NFTs, and proof is
the Merkle proof for the weights of each NFT. The function returns three
values: netInputAmount, feeAmount, and protocolFeeAmount, which
represent the total amount of base tokens spent, the amount spent on
fees, and the amount spent on protocol fees, respectively.

The function starts by performing checks to ensure the validity of the
input parameters. It calculates the sum of weights of the NFTs to buy
and validates the Merkle proof. It then calculates the required net
input amount and fee amount based on the sum of weights. If the base
token is not ETH (Ethereum), the function checks that the caller sent 0
ETH and reverts if the condition is not met.

Next, the function updates the virtual reserves by adding the net input
amount minus the fee amount and protocol fee amount to the virtual base
token reserves and subtracting the sum of weights from the virtual NFT
reserves.

In the interactions section, the function calculates the sale price for
each NFT, assuming it's the same for all NFTs even if their weights
differ. It then transfers the NFTs to the caller using the ERC721
standard. If the payRoyalties flag is set, the function calculates the
royalty fee for each NFT and adds it to the total royalty fee amount.
Finally, the royalty fee amount is added to the net input amount.

If the base token is not ETH, the function transfers the base tokens
from the caller to the contract address, and if the fee amount is
greater than 0, it transfers the fee amount to the fee recipient.

\hypertarget{token-transfer-protocol}{%
\subsubsection{Token Transfer Protocol}\label{token-transfer-protocol}}

The given code snippet is a part of a smart contract that deals with the
transfer of tokens and payment of fees in a decentralized marketplace.
The contract uses the ERC20 standard for token transfers and handles
both base tokens and Ether (ETH) as payment methods.

\begin{enumerate}
\def\labelenumi{\arabic{enumi}.}
\item
  The first step is to transfer the base token from the caller
  (msg.sender) to the contract's address. This is done using the
  safeTransferFrom function of the ERC20 standard.
\item
  If a protocol fee is set, the contract transfers the fee amount to the
  factory address using the safeTransfer function of the ERC20 standard.
\item
  If the base token is not used (i.e., ETH is used for payment), the
  contract checks if the caller has sent enough ETH to cover the net
  input amount. If not, it reverts the transaction with an
  "InvalidEthAmount" error.
\item
  If a protocol fee is set for ETH payments, the contract transfers the
  fee amount to the factory address using the safeTransferETH function.
\item
  If the caller has sent more ETH than required, the contract refunds
  the excess amount to the caller using the safeTransferETH function.
\item
  If the payRoyalties flag is set, the contract iterates through the
  tokenIds array and calculates the royalty fee for each NFT using the
  \_getRoyalty function. If the royalty fee is greater than 0 and the
  recipient address is valid, the contract transfers the royalty fee to
  the recipient. This is done using the safeTransfer function of the
  ERC20 standard for base tokens and the safeTransferETH function for
  ETH payments.
\item
  Finally, the contract emits a "Buy" event with the relevant
  information, including tokenIds, tokenWeights, netInputAmount,
  feeAmount, protocolFeeAmount, and royaltyFeeAmount. This event can be
  used by external applications to track and monitor the transactions
  happening within the contract.
\end{enumerate}

\hypertarget{nft-pool-sale}{%
\subsubsection{NFT Pool Sale}\label{nft-pool-sale}}

The \texttt{sell} function is designed to sell Non-Fungible Tokens
(NFTs) into a pool and transfer base tokens to the caller. The NFTs are
transferred from the caller to the pool, and the net sale amount depends
on the current price, fee rate, and assigned NFT weights. It is advised
not to call this function directly unless you are aware of the
consequences; instead, use a wrapper contract that checks the minimum
output amount and reverts if the slippage is too high.

The function takes the following parameters:

\begin{itemize}
\tightlist
\item
  \texttt{tokenIds}: The token IDs of the NFTs to sell.
\item
  \texttt{tokenWeights}: The weights of the NFTs to sell.
\item
  \texttt{proof}: The Merkle proof for the weights of each NFT to sell.
\item
  \texttt{stolenNftProofs}: The proofs that show each NFT is not stolen.
\end{itemize}

The function returns the following values:

\begin{itemize}
\tightlist
\item
  \texttt{netOutputAmount}: The amount of base tokens received inclusive
  of fees.
\item
  \texttt{feeAmount}: The amount of base tokens to pay in fees.
\end{itemize}

The function first calculates the sum of weights of the NFTs to sell and
validates the Merkle proof. It then calculates the net output amount and
fee amount. If the \texttt{useStolenNftOracle} flag is set, it checks
that the NFTs are not stolen using the \texttt{IStolenNftOracle}
interface.

Next, the function updates the virtual reserves by subtracting the net
output amount, protocol fee amount, and fee amount from the virtual base
token reserves and adding the weight sum to the virtual NFT reserves.

For each NFT in the \texttt{tokenIds} array, the function transfers the
NFT from the caller to the contract address. If the
\texttt{payRoyalties} flag is set, it calculates the sale price for each
NFT, gets the royalty fee and recipient, and transfers the royalty fee
to the recipient if it is greater than 0 and the recipient is not the
zero address.

The function then subtracts the royalty fee amount from the net output
amount and transfers the base tokens (either ERC20 tokens or ETH) to the
caller. If the protocol fee is set, it transfers the protocol fee to the
factory address.

Finally, the function emits a \texttt{Sell} event with the token IDs,
token weights, net output amount, fee amount, protocol fee amount, and
royalty fee amount.

\hypertarget{nft-pool-change}{%
\subsubsection{NFT Pool Change}\label{nft-pool-change}}

The \texttt{change} function allows a user to swap a set of NFTs they
own for another set of NFTs in the pool. The user must first approve the
pool to transfer their NFTs. The sum of the user's NFT weights must be
less than or equal to the sum of the output pool NFTs weights.
Additionally, the user must pay a fee based on the net input weight and
change fee amount.

The function takes the following parameters:

\begin{itemize}
\tightlist
\item
  \texttt{inputTokenIds}: The token IDs of the NFTs to change.
\item
  \texttt{inputTokenWeights}: The weights of the NFTs to change.
\item
  \texttt{inputProof}: The Merkle proof for the weights of each NFT to
  change.
\item
  \texttt{stolenNftProofs}: The proofs that show each input NFT is not
  stolen.
\item
  \texttt{outputTokenIds}: The token IDs of the NFTs to receive.
\item
  \texttt{outputTokenWeights}: The weights of the NFTs to receive.
\item
  \texttt{outputProof}: The Merkle proof for the weights of each NFT to
  receive.
\end{itemize}

The function first checks if the base token is not ETH and if the caller
sent 0 ETH. It then checks if the NFTs are not stolen using the
\texttt{IStolenNftOracle} interface. Next, it calculates the sum of
weights for the input and output NFTs and validates the proofs. If the
input weights are less than the output weights, the function reverts
with an \texttt{InsufficientInputWeight} error.

The fee amount and protocol fee amount are calculated using the
\texttt{changeFeeQuote} function. If the base token is not ETH, the
function transfers the fee amount of base tokens from the caller and the
protocol fee to the factory. If the base token is ETH, it checks if the
caller sent enough ETH to cover the fees and transfers the protocol fee
to the factory. Any excess ETH is refunded to the caller.

The function then transfers the input NFTs from the caller and the
output NFTs to the caller. Finally, it emits a \texttt{Change} event
with the input and output token IDs, weights, and fee amounts.

\hypertarget{execute-target-transaction}{%
\subsubsection{Execute Target
Transaction}\label{execute-target-transaction}}

The given code snippet defines a function called \texttt{execute} in a
smart contract, which is responsible for executing a transaction from
the pool account to a target contract. The function can only be called
by the owner of the pool, which is enforced by the \texttt{onlyOwner}
modifier. This function is useful for scenarios such as claiming
airdrops.

The \texttt{execute} function takes two input parameters:
\texttt{target}, which is the address of the target contract, and
\texttt{data}, which is the data to be sent to the target contract. The
function is also marked as \texttt{payable}, allowing it to receive
Ether during the transaction. The function returns a
\texttt{bytes\ memory} variable called \texttt{returnData}, which
contains the return data of the transaction.

Inside the function, a low-level \texttt{call} is made to the target
contract with the specified value and data. The \texttt{call} returns a
boolean \texttt{success} and a \texttt{bytes\ memory} variable
\texttt{returnData}. If the call is successful, the function returns the
\texttt{returnData}. If the call fails, the function checks if there is
any error message in the \texttt{returnData}. If there is an error
message, it bubbles up the error message using inline assembly and
reverts the transaction. If there is no error message, the transaction
is simply reverted without any error message.

\hypertarget{deposit-tokens-nfts}{%
\subsubsection{Deposit Tokens NFTs}\label{deposit-tokens-nfts}}

The given code snippet is a function named \texttt{deposit} that allows
users to deposit base tokens and NFTs (Non-Fungible Tokens) into a pool.
The function takes two parameters: an array of token IDs
(\texttt{tokenIds}) representing the NFTs to be deposited, and a
\texttt{baseTokenAmount} representing the amount of base tokens to be
deposited.

Before executing the deposit, the function checks if the base token is
ETH (Ethereum) and if the sent ETH amount (\texttt{msg.value}) is equal
to the specified \texttt{baseTokenAmount}. If the base token is not ETH,
it checks if the sent ETH amount is 0. If these conditions are not met,
the function reverts with an \texttt{InvalidEthAmount} error.

The function then proceeds to transfer the NFTs from the caller to the
pool by iterating through the \texttt{tokenIds} array and using the
\texttt{safeTransferFrom} function of the ERC721 standard. If the base
token is not ETH, it transfers the base tokens from the caller to the
pool using the \texttt{safeTransferFrom} function of the ERC20 standard.

Finally, the function emits a \texttt{Deposit} event with the deposited
NFTs and base token amount as its parameters.

It is important to note that the function is marked with a \texttt{@dev}
comment, warning developers not to call this function directly unless
they know what they are doing. Instead, they should use a wrapper
contract that checks if the current price is within the desired bounds.

\hypertarget{withdraw-nft-tokens}{%
\subsubsection{Withdraw NFT Tokens}\label{withdraw-nft-tokens}}

The given code snippet is a function named "withdraw" that allows the
owner of a pool to withdraw Non-Fungible Tokens (NFTs) and tokens from
the pool. The function takes four input parameters: the address of the
NFT (\_nft), an array of token IDs (tokenIds) representing the NFTs to
be withdrawn, the address of the token (token) to be withdrawn, and the
amount of tokens (tokenAmount) to be withdrawn.

The function is marked as "public" and can only be called by the owner
of the pool, as indicated by the "onlyOwner" modifier.

The function starts by transferring the specified NFTs to the caller
(msg.sender) using a for loop that iterates through the tokenIds array.
It does this by calling the "safeTransferFrom" function of the ERC721
contract, passing the address of the NFT, the caller's address, and the
current token ID in the loop.

Next, the function checks if the token address is equal to the zero
address (address(0)). If it is, the function transfers the specified
amount of Ether (ETH) to the caller using the "safeTransferETH"
function. If the token address is not equal to the zero address, the
function transfers the specified amount of tokens to the caller using
the "transfer" function of the ERC20 contract.

Finally, the function emits a "Withdraw" event with the NFT address,
token IDs array, token address, and token amount as its parameters. This
event can be used by external applications to track and monitor the
withdrawal of NFTs and tokens from the pool.

\hypertarget{set-virtual-reserves}{%
\subsubsection{Set Virtual Reserves}\label{set-virtual-reserves}}

The given code snippet is a Solidity function called
\texttt{setVirtualReserves} that sets the virtual base token reserves
and virtual NFT (Non-Fungible Token) reserves for a pool. This function
can only be called by the owner of the pool. The parameters of this
function are \texttt{newVirtualBaseTokenReserves} and
\texttt{newVirtualNftReserves}, which are both of type \texttt{uint128}.
These parameters affect the price and liquidity depth of the pool.

The function first sets the \texttt{virtualBaseTokenReserves} and
\texttt{virtualNftReserves} variables to the values of the input
parameters \texttt{newVirtualBaseTokenReserves} and
\texttt{newVirtualNftReserves}, respectively. After updating the reserve
values, the function emits an event called \texttt{SetVirtualReserves}
with the new reserve values as its arguments. This event can be used by
external systems to track changes in the virtual reserves of the pool.

\hypertarget{set-merkle-root}{%
\subsubsection{Set Merkle Root}\label{set-merkle-root}}

The given code snippet is a Solidity function named
\texttt{setMerkleRoot} that is used to update the Merkle root of a pool.
This function can only be called by the owner of the pool, as indicated
by the \texttt{onlyOwner} modifier. The Merkle root is utilized for
validating the weights of Non-Fungible Tokens (NFTs) within the pool.

The function takes a single input parameter, \texttt{newMerkleRoot},
which is of type \texttt{bytes32}. This parameter represents the new
Merkle root value that will replace the current one.

Inside the function, the Merkle root is updated by assigning the value
of \texttt{newMerkleRoot} to the \texttt{merkleRoot} variable. Following
this, an event named \texttt{SetMerkleRoot} is emitted with the updated
\texttt{newMerkleRoot} value as its argument. This event allows external
entities to listen for changes in the Merkle root and react accordingly.

\hypertarget{fee-rate-setter}{%
\subsubsection{Fee Rate Setter}\label{fee-rate-setter}}

The given code snippet is a function called \texttt{setFeeRate} that
sets the fee rate for a pool. This function can only be called by the
owner of the pool. The fee rate is used to calculate the fee amount when
swapping or changing NFTs (Non-Fungible Tokens). The fee rate is
expressed in basis points, where 1 basis point is equal to 1/100th of a
percent. For instance, a fee rate of 10,000 basis points represents
100\%, 200 basis points represent 2\%, and 1 basis point represents
0.01\%.

The function takes a single parameter, \texttt{newFeeRate}, which is the
new fee rate to be set, in basis points. Inside the function, there is a
check to ensure that the new fee rate is less than 50\% (5,000 basis
points). If the new fee rate is greater than 5,000 basis points, the
function reverts with a \texttt{FeeRateTooHigh} error.

If the new fee rate is valid, the function sets the fee rate to the new
value and emits an event called \texttt{SetFeeRate} with the new fee
rate as its parameter. This event can be used by external systems to
track changes in the fee rate.

\hypertarget{set-pay-royalties}{%
\subsubsection{Set Pay Royalties}\label{set-pay-royalties}}

The given code snippet is a Solidity function named
\texttt{setPayRoyalties} that is used to set the pay royalties flag in a
smart contract. This function can only be called by the owner of the
pool, as indicated by the \texttt{onlyOwner} modifier.

The function takes a single input parameter, \texttt{newPayRoyalties},
which is a boolean value representing the new pay royalties flag. When
royalties are enabled (i.e., the flag is set to \texttt{true}), the pool
will pay royalties when buying or selling NFTs (Non-Fungible Tokens).

Inside the function, the pay royalties flag is updated with the new
value provided by the \texttt{newPayRoyalties} parameter. After updating
the flag, an event named \texttt{SetPayRoyalties} is emitted with the
new pay royalties flag value. This event can be used by external systems
or applications to track changes in the pay royalties flag.

\hypertarget{update-parameter-settings}{%
\subsubsection{Update Parameter
Settings}\label{update-parameter-settings}}

The given code snippet is a function called \texttt{setAllParameters} in
a smart contract, which allows updating multiple parameter settings in a
single function call. The function takes six input parameters:

\begin{enumerate}
\def\labelenumi{\arabic{enumi}.}
\tightlist
\item
  \texttt{newVirtualBaseTokenReserves}: The new virtual base token
  reserves value (uint128).
\item
  \texttt{newVirtualNftReserves}: The new virtual NFT reserves value
  (uint128).
\item
  \texttt{newMerkleRoot}: The new Merkle root value (bytes32).
\item
  \texttt{newFeeRate}: The new fee rate value (uint16) in basis points.
\item
  \texttt{newUseStolenNftOracle}: The new flag (bool) indicating whether
  to use a stolen NFT oracle or not.
\item
  \texttt{newPayRoyalties}: The new flag (bool) indicating whether to
  pay royalties or not.
\end{enumerate}

The function then calls five other functions to update the respective
parameters:

\begin{enumerate}
\def\labelenumi{\arabic{enumi}.}
\tightlist
\item
  \texttt{setVirtualReserves(newVirtualBaseTokenReserves,\ newVirtualNftReserves)}:
  Updates the virtual base token and NFT reserves with the new values.
\item
  \texttt{setMerkleRoot(newMerkleRoot)}: Updates the Merkle root with
  the new value.
\item
  \texttt{setFeeRate(newFeeRate)}: Updates the fee rate with the new
  value.
\item
  \texttt{setUseStolenNftOracle(newUseStolenNftOracle)}: Updates the
  flag for using a stolen NFT oracle with the new value.
\item
  \texttt{setPayRoyalties(newPayRoyalties)}: Updates the flag for paying
  royalties with the new value.
\end{enumerate}

This function provides a convenient way to update multiple parameter
settings in a single transaction, which can save gas costs and simplify
the process for users.

\hypertarget{flash-loan-execution}{%
\subsubsection{Flash Loan Execution}\label{flash-loan-execution}}

The given code snippet is a function called \texttt{flashLoan} that
executes a flash loan in a smart contract. The function takes four
parameters: \texttt{receiver}, \texttt{token}, \texttt{tokenId}, and
\texttt{data}. The \texttt{receiver} is the address of the entity
receiving the flash loan, \texttt{token} is the address of the
Non-Fungible Token (NFT) contract, \texttt{tokenId} is the ID of the
NFT, and \texttt{data} is any additional data to be passed to the
receiver.

The function is marked as \texttt{external}, meaning it can only be
called from outside the contract, and \texttt{payable}, allowing it to
receive Ether (ETH) as payment. The function returns a boolean value
indicating the success of the flash loan.

First, the function checks if the NFT is available for a flash loan
using the \texttt{availableForFlashLoan} function. If it is not
available, the function reverts with a \texttt{NotAvailableForFlashLoan}
error.

Next, the function calculates the fee for the flash loan using the
\texttt{flashFee} function. If the base token is ETH (i.e.,
\texttt{baseToken} is the zero address), the function checks if the
caller sent enough ETH to cover the fee. If not, it reverts with an
\texttt{InvalidEthAmount} error.

The function then transfers the NFT to the borrower using the
\texttt{safeTransferFrom} function of the ERC721 contract. It calls the
borrower's \texttt{onFlashLoan} function, passing the necessary
parameters, and checks if the flash loan was successful by comparing the
returned value with the expected hash of the
"ERC3156FlashBorrower.onFlashLoan" string. If the flash loan was not
successful, the function reverts with a \texttt{FlashLoanFailed} error.

After a successful flash loan, the function transfers the NFT back from
the borrower using the \texttt{safeTransferFrom} function of the ERC721
contract. If the base token is not ETH, it transfers the fee from the
borrower using the \texttt{transferFrom} function of the ERC20 contract.

Finally, the function returns the \texttt{success} boolean value,
indicating whether the flash loan was successful or not.

\hypertarget{sum-validate-proof}{%
\subsubsection{Sum Validate Proof}\label{sum-validate-proof}}

The given code snippet is a Solidity function called
\texttt{sumWeightsAndValidateProof} that takes three input parameters:
an array of token IDs (\texttt{tokenIds}), an array of corresponding
token weights (\texttt{tokenWeights}), and a Merkle multi-proof object
(\texttt{proof}). The function's purpose is to calculate the sum of the
weights of each NFT (Non-Fungible Token) and validate that the weights
are correct by verifying the provided Merkle proof.

Initially, the function checks if the \texttt{merkleRoot} is not set
(i.e., equals to \texttt{bytes32(0)}). If this is the case, it sets the
weight of each NFT to be \texttt{1e18} and returns the product of the
number of token IDs and this weight.

If the \texttt{merkleRoot} is set, the function initializes a variable
\texttt{sum} to store the sum of the token weights and creates a new
array \texttt{leafs} to store the Merkle proof leaves. It then iterates
through the \texttt{tokenIds} array and, for each token ID, creates a
leaf by hashing the concatenated keccak256 hash of the token ID and its
corresponding weight. The function also adds the token weight to the
\texttt{sum} variable.

After iterating through all token IDs, the function verifies the Merkle
proof using the \texttt{MerkleProofLib.verifyMultiProof} function. If
the proof is not valid, it reverts the transaction with an
\texttt{InvalidMerkleProof} error. If the proof is valid, the function
returns the calculated sum of the token weights.

\hypertarget{nft-buy-quote}{%
\subsubsection{NFT Buy Quote}\label{nft-buy-quote}}

The given code snippet is a function called \texttt{buyQuote} in a smart
contract, which calculates the required input amount of base tokens
needed to buy a specified amount of NFTs (Non-Fungible Tokens), along
with the associated fee amounts.

The function takes a single input parameter, \texttt{outputAmount},
which represents the desired amount of NFTs to buy, multiplied by 1e18
for precision. It returns three values: \texttt{netInputAmount},
\texttt{feeAmount}, and \texttt{protocolFeeAmount}.

The function first calculates the input amount using the xy=k invariant,
a formula used in automated market makers (AMMs) to maintain a constant
product of token reserves. It does this by calling the \texttt{mulDivUp}
function from the \texttt{FixedPointMathLib} library, passing the
\texttt{outputAmount}, \texttt{virtualBaseTokenReserves}, and the
difference between \texttt{virtualNftReserves} and
\texttt{outputAmount}. The result is rounded up by 1 wei (the smallest
unit of Ether) for precision.

Next, the function calculates the protocol fee amount by multiplying the
input amount by the protocol fee rate, which is fetched from the factory
contract using the \texttt{protocolFeeRate()} function. The result is
divided by 10,000 to get the actual fee amount.

Similarly, the function calculates the fee amount by multiplying the
input amount by the \texttt{feeRate} and dividing the result by 10,000.

Finally, the function calculates the net input amount by adding the
input amount, fee amount, and protocol fee amount together. This net
input amount represents the total amount of base tokens required to buy
the specified amount of NFTs, inclusive of all fees.

\hypertarget{nft-sell-quote}{%
\subsubsection{NFT Sell Quote}\label{nft-sell-quote}}

The \texttt{sellQuote} function is a public view function that returns
the output amount of selling a given amount of NFTs (Non-Fungible
Tokens) inclusive of the fee, which depends on the currently set fee
rate. The function takes one input parameter, \texttt{inputAmount},
which represents the amount of NFTs to sell multiplied by 1e18. The
function returns three values: \texttt{netOutputAmount},
\texttt{feeAmount}, and \texttt{protocolFeeAmount}.

The function calculates the output amount based on the xy=k invariant, a
formula used in automated market makers (AMMs) to maintain a constant
product of token reserves. The output amount is calculated as
\texttt{inputAmount\ *\ virtualBaseTokenReserves\ /\ (virtualNftReserves\ +\ inputAmount)}.

Next, the function calculates the protocol fee amount by multiplying the
output amount by the protocol fee rate obtained from the Factory
contract and dividing by 10,000. The fee amount is calculated similarly,
by multiplying the output amount by the fee rate and dividing by 10,000.

Finally, the function calculates the net output amount by subtracting
the fee amount and protocol fee amount from the output amount. The
function then returns the net output amount, fee amount, and protocol
fee amount.

\hypertarget{nft-change-fee}{%
\subsubsection{NFT Change Fee}\label{nft-change-fee}}

The \texttt{changeFeeQuote} function calculates and returns the fee
required to change a specified amount of NFTs (Non-Fungible Tokens)
based on the current \texttt{changeFee}. The function takes
\texttt{inputAmount} as a parameter, which represents the amount of NFTs
to change, multiplied by 1e18. The function returns two values:
\texttt{feeAmount} and \texttt{protocolFeeAmount}.

First, the function calculates the exponent based on the
\texttt{baseToken} decimals. If the \texttt{baseToken} is the zero
address, the exponent is set to 14 (18 - 4). Otherwise, the exponent is
calculated as the difference between the \texttt{baseToken} decimals and
4. The \texttt{changeFee} is then multiplied by 10 raised to the power
of the exponent to obtain the fee per NFT with 4 decimals of accuracy.

Next, the \texttt{feeAmount} is calculated by multiplying the
\texttt{inputAmount} by the \texttt{feePerNft} and dividing the result
by 1e18. The \texttt{protocolFeeAmount} is then calculated by
multiplying the \texttt{feeAmount} by the \texttt{protocolFeeRate} from
the Factory contract and dividing the result by 10,000.

In summary, the \texttt{changeFeeQuote} function calculates the fee and
protocol fee amounts required to change a specified amount of NFTs based
on the current \texttt{changeFee} and the \texttt{baseToken} decimals.

\hypertarget{pool-price-function}{%
\subsubsection{Pool Price Function}\label{pool-price-function}}

The given code snippet is a Solidity function named \texttt{price()}
that returns the price of a pool with 18 decimals of accuracy. The
function is marked as \texttt{public} and \texttt{view}, meaning it can
be called by anyone and does not modify the state of the contract.

The function calculates the price by first determining the exponent to
be used for scaling the result. This is done by checking if the
\texttt{baseToken} address is equal to the zero address (i.e., no base
token is set). If this is the case, the exponent is set to 18, ensuring
18 decimals of accuracy. Otherwise, the exponent is calculated as 36
minus the number of decimals of the \texttt{baseToken} (retrieved using
the ERC20 standard's \texttt{decimals()} function).

Next, the function calculates the price by multiplying the
\texttt{virtualBaseTokenReserves} by 10 raised to the power of the
previously calculated exponent. This result is then divided by the
\texttt{virtualNftReserves} to obtain the final price.

Finally, the function returns the calculated price as a \texttt{uint256}
value.

\hypertarget{flash-swap-fee}{%
\subsubsection{Flash Swap Fee}\label{flash-swap-fee}}

The given code snippet is a Solidity function named \texttt{flashFee}
that is part of a smart contract. The purpose of this function is to
return the fee required to perform a flash swap of a specific
Non-Fungible Token (NFT).

The function has the following characteristics:

\begin{enumerate}
\def\labelenumi{\arabic{enumi}.}
\item
  Visibility: The function is marked as \texttt{public}, which means it
  can be called from any external address or from within the smart
  contract itself.
\item
  State mutability: The function is marked as \texttt{view}, which
  indicates that it does not modify the state of the contract. It only
  reads the state and returns a value.
\item
  Parameters: The function accepts two parameters - an \texttt{address}
  and a \texttt{uint256}. The \texttt{address} parameter represents the
  address of the NFT, and the \texttt{uint256} parameter represents the
  unique identifier of the NFT.
\item
  Return value: The function returns a single value of type
  \texttt{uint256}, which represents the fee amount required to perform
  the flash swap of the given NFT.
\item
  Function logic: The function simply returns the value of a variable
  named \texttt{changeFee}. This variable is assumed to be a state
  variable within the smart contract that holds the fee amount for flash
  swaps. The function does not perform any calculations or logic beyond
  returning this value.
\end{enumerate}

In summary, the \texttt{flashFee} function is a public view function
that takes an address and a unique identifier as input parameters and
returns the fee amount required to flash swap a specific NFT by simply
returning the value of a state variable named \texttt{changeFee}.

\hypertarget{flash-fee-token}{%
\subsubsection{Flash Fee Token}\label{flash-fee-token}}

The given content is a Solidity function named \texttt{flashFeeToken()}
within a smart contract. This function is marked as \texttt{public} and
\texttt{view}, meaning it can be called by anyone and does not modify
the contract's state. The purpose of this function is to return the
address of the token that is used to pay the flash fee.

The function has no input parameters and returns a single output of type
\texttt{address}. Inside the function body, the \texttt{baseToken}
variable is returned as the result. This implies that the
\texttt{baseToken} represents the token used for paying flash fees in
the context of this smart contract.

\hypertarget{nft-flash-loan}{%
\subsubsection{NFT Flash Loan}\label{nft-flash-loan}}

The given content is a Solidity function named
\texttt{availableForFlashLoan} that checks the availability of a
Non-Fungible Token (NFT) for a flash loan. The function takes two input
parameters: \texttt{token}, which is the address of the NFT contract,
and \texttt{tokenId}, which is the unique identifier of the NFT. The
function returns a boolean value \texttt{available}, indicating whether
the specified NFT is available for a flash loan or not.

The function is marked as \texttt{public}, meaning it can be called from
any external contract or account, and \texttt{view}, which indicates
that it does not modify the state of the contract. This function is
designed to be used in the context of decentralized finance (DeFi)
applications, where flash loans are a common feature. By providing a way
to check the availability of an NFT for a flash loan, this function can
help ensure that only eligible NFTs are used in flash loan transactions.

\hypertarget{nft-ownership-check}{%
\subsubsection{NFT Ownership Check}\label{nft-ownership-check}}

The given code snippet is a function that checks if a specific
Non-Fungible Token (NFT) is owned by the current smart contract. It does
this by interacting with the ERC721 standard, which is a widely used
standard for creating and managing NFTs on the Ethereum blockchain.

\begin{enumerate}
\def\labelenumi{\arabic{enumi}.}
\item
  The function starts by attempting to call the \texttt{ownerOf}
  function from the ERC721 smart contract, which is identified by the
  \texttt{token} address. The \texttt{ownerOf} function takes a
  \texttt{tokenId} as its argument and returns the address of the
  current owner of the NFT with the specified \texttt{tokenId}.
\item
  The \texttt{try} keyword is used to handle any potential errors that
  may occur during the execution of the \texttt{ownerOf} function call.
  If the function call is successful, the returned owner address is
  stored in the \texttt{result} variable.
\item
  The function then checks if the \texttt{result} (owner address) is
  equal to the address of the current smart contract (denoted by
  \texttt{address(this)}). If the addresses match, it means that the NFT
  is owned by the current smart contract, and the function returns
  \texttt{true}. If the addresses do not match, the function returns
  \texttt{false}.
\item
  If an error occurs during the execution of the \texttt{ownerOf}
  function call, the \texttt{catch} block is executed. In this case, the
  function returns \texttt{false}, indicating that the NFT is not owned
  by the current smart contract.
\end{enumerate}

In summary, this function checks if a specific NFT, identified by its
\texttt{tokenId}, is owned by the current smart contract by interacting
with the ERC721 standard. It returns \texttt{true} if the NFT is owned
by the contract and \texttt{false} otherwise.

\hypertarget{royalty-fee-lookup}{%
\subsubsection{Royalty Fee Lookup}\label{royalty-fee-lookup}}

The given content is a function definition in a smart contract, written
in Solidity programming language. The function is named
\texttt{\_getRoyalty} and is used to calculate the royalty fee and
recipient for a given Non-Fungible Token (NFT) and its sale price. The
function fetches the royalty information from the Manifold registry.

The function takes two input parameters:

\begin{enumerate}
\def\labelenumi{\arabic{enumi}.}
\tightlist
\item
  \texttt{tokenId}: A \texttt{uint256} data type representing the unique
  identifier of the NFT.
\item
  \texttt{salePrice}: A \texttt{uint256} data type representing the sale
  price of the NFT.
\end{enumerate}

The function returns two output values:

\begin{enumerate}
\def\labelenumi{\arabic{enumi}.}
\tightlist
\item
  \texttt{royaltyFee}: A \texttt{uint256} data type representing the
  royalty fee to be paid.
\item
  \texttt{recipient}: An \texttt{address} data type representing the
  address to which the royalty fee should be paid.
\end{enumerate}

The function is marked as \texttt{internal}, which means it can only be
called from within the same contract or contracts derived from it. It is
also marked as \texttt{view}, indicating that it does not modify the
state of the contract and only reads from it.

Inside the function, the royalty lookup address is fetched by calling
the \texttt{getRoyaltyLookupAddress} function of the
\texttt{IRoyaltyRegistry} interface, passing the \texttt{nft} address as
an argument. The \texttt{IRoyaltyRegistry} interface is a contract that
defines the functions for interacting with the Manifold registry. The
\texttt{royaltyRegistry} variable is an instance of this interface, and
it is used to interact with the Manifold registry to fetch the royalty
information.

In summary, the \texttt{\_getRoyalty} function is a utility function in
a smart contract that calculates the royalty fee and recipient for a
given NFT and its sale price by fetching the royalty information from
the Manifold registry.

\hypertarget{royalty-fee-validation-1}{%
\subsubsection{Royalty Fee Validation}\label{royalty-fee-validation-1}}

The given code snippet is written in Solidity, a programming language
used for implementing smart contracts on the Ethereum blockchain. It
checks if a specific contract supports the ERC-2981 interface, which is
a standard for handling royalties in Non-Fungible Tokens (NFTs). If the
contract supports this interface, the code retrieves the royalty
information and ensures that the royalty fee is not greater than the
sale price of the NFT.

\begin{enumerate}
\def\labelenumi{\arabic{enumi}.}
\item
  The first line checks if the contract at the \texttt{lookupAddress}
  supports the ERC-2981 interface by calling the
  \texttt{supportsInterface} function with the interface ID of the
  ERC-2981 standard. The \texttt{type(IERC2981).interfaceId} expression
  retrieves the interface ID of the ERC-2981 standard.
\item
  If the contract supports the ERC-2981 interface, the code proceeds to
  retrieve the royalty information for the given \texttt{tokenId} and
  \texttt{salePrice}. The \texttt{royaltyInfo} function is called on the
  contract at the \texttt{lookupAddress}, and the returned values are
  stored in the \texttt{recipient} and \texttt{royaltyFee} variables.
\item
  The code then checks if the \texttt{royaltyFee} is greater than the
  \texttt{salePrice}. If this condition is true, the transaction is
  reverted with an \texttt{InvalidRoyaltyFee} error message. This
  ensures that the royalty fee cannot exceed the sale price of the NFT.
\item
  The code snippet is enclosed within a function or a contract, as
  indicated by the closing curly braces. However, the context and
  purpose of the enclosing function or contract are not provided in the
  given content.
\end{enumerate}

\hypertarget{spearbot-nodeput_files_to_audit_heresolidityprivatepoolmetadatasol}{%
\subsection{spearbot-node/put\_files\_to\_audit\_here/solidity/PrivatePoolMetadata.sol}\label{spearbot-nodeput_files_to_audit_heresolidityprivatepoolmetadatasol}}

Summary: The PrivatePoolMetadata contract generates NFT metadata for
private pools. It includes functions to return the tokenURI with its
metadata, attributes encoded as JSON, and an SVG image for a pool, given
the private pool's token ID. The contract imports and utilizes various
libraries such as Strings, Base64, ERC20, and ERC721.

\hypertarget{private-pool-metadata-1}{%
\subsubsection{Private Pool Metadata}\label{private-pool-metadata-1}}

The Private Pool Metadata contract, authored by out.eth (@outdoteth), is
designed to generate Non-Fungible Token (NFT) metadata for private
pools. NFTs are unique digital assets that can represent various items
such as art, collectibles, and virtual real estate. Metadata is the
information that describes the attributes and properties of these NFTs.

In the context of this contract, private pools refer to exclusive
liquidity pools where access is restricted to a specific group of users.
These pools can be used for various purposes, such as private sales,
pre-sales, or exclusive access to certain assets.

The contract contains functions and data structures that facilitate the
creation and management of NFT metadata for these private pools. Some of
the key components of the contract include:

\begin{enumerate}
\def\labelenumi{\arabic{enumi}.}
\item
  Structs: The contract defines a struct called \texttt{Metadata}, which
  holds the information related to an NFT's metadata. This includes
  attributes such as the token's name, description, image URL, and other
  relevant details.
\item
  Mappings: The contract uses a mapping to associate each NFT's unique
  identifier (token ID) with its corresponding metadata. This allows for
  efficient retrieval and updating of metadata for a specific NFT.
\item
  Functions: The contract contains various functions that enable users
  to interact with the metadata. Some of these functions include:

  \begin{itemize}
  \item
    \texttt{mint}: This function allows the contract owner to create a
    new NFT with the specified metadata. It takes the token ID and
    metadata as input parameters and updates the mapping accordingly.
  \item
    \texttt{updateMetadata}: This function enables the contract owner to
    update the metadata of an existing NFT. It takes the token ID and
    new metadata as input parameters and updates the mapping
    accordingly.
  \item
    \texttt{getMetadata}: This function allows users to retrieve the
    metadata of a specific NFT by providing its token ID. It returns the
    metadata as an output parameter.
  \item
    \texttt{tokenURI}: This function returns a URI that points to the
    metadata of a specific NFT. This is a standard function required by
    the ERC721 NFT standard and is used by various platforms and
    marketplaces to display the NFT's metadata.
  \end{itemize}
\end{enumerate}

In summary, the Private Pool Metadata contract provides a robust and
efficient solution for managing NFT metadata in the context of private
pools. By leveraging the features of this contract, developers can
create exclusive and customized NFT experiences for their users.

\hypertarget{pool-tokenuri-metadata}{%
\subsubsection{Pool TokenURI Metadata}\label{pool-tokenuri-metadata}}

The given code snippet is a Solidity function called \texttt{tokenURI}
that takes a \texttt{tokenId} as input and returns a string containing
the tokenURI for a private pool with its metadata. The function is
marked as \texttt{public\ view}, meaning it can be called by anyone and
does not modify the contract's state.

The function starts by creating a \texttt{bytes} variable called
\texttt{metadata} and encoding a JSON object as a packed byte array
using \texttt{abi.encodePacked}. The JSON object contains the following
properties:

\begin{enumerate}
\def\labelenumi{\arabic{enumi}.}
\tightlist
\item
  "name": A string that concatenates "Private Pool " with the
  \texttt{tokenId} converted to a string using
  \texttt{Strings.toString(tokenId)}.
\item
  "description": A static string "Caviar private pool AMM position."
\item
  "image": A data URI containing an SVG image encoded in base64 format.
  The SVG image is generated by calling the \texttt{svg(tokenId)}
  function and then encoded using \texttt{Base64.encode}.
\item
  "attributes": An array of attributes for the token, obtained by
  calling the \texttt{attributes(tokenId)} function.
\end{enumerate}

Finally, the function returns the tokenURI as a string by concatenating
"data:application/json;base64," with the base64-encoded
\texttt{metadata} using \texttt{abi.encodePacked} and
\texttt{Base64.encode}. The resulting string is a data URI containing
the JSON metadata for the private pool token in base64 format.

\hypertarget{pool-attributes-function}{%
\subsubsection{Pool Attributes
Function}\label{pool-attributes-function}}

The given code snippet is a Solidity function named \texttt{attributes}
that takes a \texttt{tokenId} as input and returns a JSON-encoded string
containing various attributes of a private pool. The function is marked
as \texttt{public\ view}, meaning it can be called by anyone and does
not modify the state of the contract.

First, the function creates a \texttt{PrivatePool} instance by casting
the \texttt{tokenId} to an address and then to a \texttt{PrivatePool}
object. Next, it encodes the attributes of the private pool into a bytes
array called \texttt{\_attributes} using the \texttt{abi.encodePacked}
function. The attributes included are:

\begin{enumerate}
\def\labelenumi{\arabic{enumi}.}
\tightlist
\item
  Pool address: The address of the private pool.
\item
  Base token: The address of the base token used in the pool.
\item
  NFT: The address of the NFT used in the pool.
\item
  Virtual base token reserves: The virtual reserves of the base token in
  the pool.
\item
  Virtual NFT reserves: The virtual reserves of the NFT in the pool.
\item
  Fee rate (bps): The fee rate of the pool in basis points.
\item
  NFT balance: The balance of the NFT held by the private pool.
\item
  Base token balance: The balance of the base token held by the private
  pool.
\end{enumerate}

Finally, the function returns the \texttt{\_attributes} bytes array as a
string.

\hypertarget{svg-image-generator}{%
\subsubsection{Svg Image Generator}\label{svg-image-generator}}

The given code snippet is a Solidity function named \texttt{svg} that
takes a \texttt{tokenId} as input and returns an SVG image in bytes
format. The function is designed to generate an SVG image containing
information about a private pool in an Automated Market Maker (AMM)
system called "Caviar AMM."

The function first creates a \texttt{PrivatePool} object using the
provided \texttt{tokenId}. It then constructs the SVG image in three
separate scopes to avoid "stack too deep" errors. The SVG image is built
using the \texttt{abi.encodePacked} function, which concatenates the
input arguments into a single bytes array.

In the first scope, the SVG image is initialized with a black
background, white fill, and a serif font. It includes the following text
elements:

\begin{enumerate}
\def\labelenumi{\arabic{enumi}.}
\tightlist
\item
  "Caviar AMM private pool position"
\item
  "Private pool: " followed by the address of the \texttt{privatePool}
  object
\item
  "Base token: " followed by the address of the base token in the
  \texttt{privatePool}
\item
  "NFT: " followed by the address of the NFT in the \texttt{privatePool}
\end{enumerate}

In the second scope, the SVG image is updated with additional text
elements:

\begin{enumerate}
\def\labelenumi{\arabic{enumi}.}
\setcounter{enumi}{4}
\tightlist
\item
  "Virtual base token reserves: " followed by the virtual base token
  reserves of the \texttt{privatePool}
\item
  "Virtual NFT reserves: " followed by the virtual NFT reserves of the
  \texttt{privatePool}
\item
  "Fee rate (bps): " followed by the fee rate of the
  \texttt{privatePool}
\end{enumerate}

In the third and final scope, the SVG image is updated with the last two
text elements and closed with the \texttt{\textless{}/svg\textgreater{}}
tag:

\begin{enumerate}
\def\labelenumi{\arabic{enumi}.}
\setcounter{enumi}{7}
\tightlist
\item
  "NFT balance: " followed by the NFT balance of the
  \texttt{privatePool}
\item
  "Base token balance: " followed by the base token balance of the
  \texttt{privatePool}
\end{enumerate}

Finally, the function returns the constructed SVG image in bytes format.

\hypertarget{trait-type-value}{%
\subsubsection{Trait Type Value}\label{trait-type-value}}

The given code snippet is a function named \texttt{trait} written in
Solidity, a programming language used for implementing smart contracts
on the Ethereum blockchain. This function takes two input parameters,
both of type \texttt{string\ memory}: \texttt{traitType} and
\texttt{value}. The function is marked as \texttt{internal}, meaning it
can only be called from within the same contract or contracts derived
from it, and \texttt{pure}, indicating that it does not modify the
contract's state or access any external data.

The purpose of this function is to generate a JSON-formatted string
representing a trait object with two properties: "trait\_type" and
"value". The values of these properties are derived from the input
parameters \texttt{traitType} and \texttt{value}, respectively.

The function returns a \texttt{string\ memory} type, which is the
JSON-formatted string created using the \texttt{abi.encodePacked}
function. The \texttt{abi.encodePacked} function is part of the Ethereum
Application Binary Interface (ABI) and is used to tightly pack the input
arguments into a single bytes sequence. In this case, the input
arguments are a combination of string literals and the input parameters
\texttt{traitType} and \texttt{value}, which together form the desired
JSON string.

The \texttt{return} statement constructs a new \texttt{string} type from
the packed bytes sequence generated by \texttt{abi.encodePacked}. The
resulting JSON-formatted string is then returned as the output of the
\texttt{trait} function.
